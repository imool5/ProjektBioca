\tableofcontents
\chapter{Wprowadzenie}

\section{Sztuczna Sie� neuronowa - co to w�a�ciwie jest?}

Definicj� sztucznej sieci neuronowej jest zbi�r prostych jednostek obliczeniowych przetwarzaj�cych dane,
komunikuj�cych si� ze sob� i pracuj�cych r�wnolegle.

\section{Typy Sieci Neuronowych}

Wyr�niamy 3 typy sieci neuronowych
\begin{itemize}
\item Sieci Jednokierunkowe
\item Sieci Rekurencyjne
\item Samoorganizuj�ce si� mapy
\end{itemize}







\begin{flushleft}
Tekst wyr�wnany do lewej.
\end{flushleft}
\begin{center}
Tekst wy�rodkowany.
\end{center}
\begin{flushright}
Tekst wyr�wnany do prawej.
\end{flushright}
\section{Podrozdzia� 1}
\section{Podrozdzia� 2}

\begin{itemize}
\item[-] \textbf{Pogrubiony tekst.}
\item \textit{Tekst pisany kursyw�.}
\end{itemize}
\begin{enumerate}
\item punkt pierwszy
\item punkt drugi
\end{enumerate}
\subsection{Znaki Specjalne}
Hasztag \# \\
Backslash $\backslash$
 Dolar $\$$

 \begin{table} [h] 
 \begin{tabular}{|l|c|p{7cm}|} 
 \hline 
 kom 11 & kom 12  \\ 
 \hline 
 \hline
\multicolumn{2}{|c|}{kom 22 i 23}  \\ 
\hline
\multicolumn{2}{|l|}{kom 31 i kom 32}   \\ 
\hline 
 kom 41 & kom 42  \\ 
 \hline 
 \end{tabular} 
 \centering 
 \caption{tabela 1}\label{tab_1} 
 
 \end{table}

\ odwo�anie do tabeli (tab \ref{tab_1}):
 \begin{verbatim}
 
\begin{table} [h] 
 \begin{tabular}{|l|c|p{7cm}|} 
 \hline 
 kom 11 & kom 12  \\ 
 \hline 
 \hline
\multicolumn{2}{|c|}{kom 22 i 23}  \\ 
\hline
\multicolumn{2}{|l|}{kom 31 i kom 32}   \\ 
\hline 
 kom 41 & kom 42  \\ 
 \hline 
 \end{tabular} 
 \centering 
 \caption{tabela 1}\label{tab_1} 
 
\end{verbatim}

\begin{equation}\label{moje_r�wnanie}
\left(\prod_{i=\widetilde{j}}^{\infty}[\log(i^{\xi})]^{M}\le0 \leftrightarrow \sum_{i=\widetilde{j}}^{\infty}\sqrt{{i}\over{\widetilde{j}}} >
\sqrt[p]{\widetilde{j}}\right )\Rightarrow \textrm{nic nie}\ wynika
\end{equation}
\newline
\ Odwo�anie do r�wnania  (\ref{moje_r�wnanie}):
\begin{verbatim}
\begin{equation}\label{moje_r�wnanie}
\left(\prod_{i=\widetilde{j}}^{\infty}[\log(i^{\xi})]^{M}\le0
 \leftrightarrow
  \sum_{i=\widetilde{j}}^{\infty}\sqrt{{i}\over{\widetilde{j}}} >
\sqrt[p]{\widetilde{j}}\right )\Rightarrow \textrm{nic nie}\ wynika
\end{equation}
\end{verbatim}



\begin{figure}
\raggedright
  
\caption{ dwa rysunki jeden nad drugim}\label{rysunek_1}

\end{figure}

\ Odwo�anie do rysunku (rys \ref{rysunek_1}):
\begin{verbatim}
\begin{figure}
\raggedright
\includegraphics[scale=0.35]{rysunek1.jpg}
\includegraphics[scale=0.35]{rysunek1.jpg}  
\caption{ dwa rysunki jeden nad drugim}\label{rysunek_1}

\end{figure}
\end{verbatim}
 \end{document}